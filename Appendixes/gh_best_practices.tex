\section{About GitHub Documentation}\label{sec:about_docs}\
At GitHub, we strive to create documentation that is accurate, valuable, inclusive, accessible, and easy to use.
Before contributing to GitHub Docs, please take a moment to familiarize yourself with GitHub's documentation philosophy, fundamentals, and content design principles, given in Sub-Sections.
\subsection{GitHub's Documentation Philosophy}\label{subsec:doc_ply}
Our documentation philosophy guides what content we create and how we create it.
As the home to the world's largest community of developers, we want to make sure that our documentation is accurate,
valuable, inclusive, and easy to use.
Our documentation philosophy leads us toward these goals.

\begin{enumerate}
    \item We advocate for our users.
    This can occur at any point in the documentation process, from planning to writing to publishing.
    We respond to feedback and proactively work to create the best user experience possible on the GitHub site.
    \item We write for an international audience, so that our content can be translated and is inclusive of all people.
    \item We create content that can be accessed by a broad group of users.
    Our content toolkit includes resources and guidance to reduce barriers for people with disabilities, and we
    prioritize keeping up to date with the latest accessibility standards.
    \item We create task-based content.
    We consider what people are trying to accomplish when they use GitHub, and then we create content that helps
    them achieve their goals and discover new possibilities.
    \item We collaborate with teams across GitHub and the open source community to create high-quality content.
    \item We continually learn and improve to curate the best experience for people using GitHub.
\end{enumerate}
%We may want examples of each or most of these points

\subsection{GitHub's Documentation Fundamentals}\label{subsec:doc_fundm}
All content published must meet a small number of fundamental requirements.
Use the lists below to help ensure your contributions are accurate, accessible and inclusive, and consistent.
\begin{enumerate}
    \item \textbf{Accurate}: Documentation must be correct and accurate.
    Ensure that the content is free from factual, spelling, and formatting errors.
    \item Documentation is up to date with the latest accessibility standards, and is written to be inclusive and
    translation-friendly.
    \begin{enumerate}
        \item Ensure content adheres to the accessibility and screenshot guidelines.
        For more information, see \href{https://docs.github.com/en/contributing/writing-for-github-docs/creating-screenshots}{Creating screenshots}, or in Appendix~\ref{ch:create_ss}, page~\pageref{ch:create_ss}.
        \item Ensure content can be successfully translated.
        For more information, see \href{https://docs.github.com/en/contributing/writing-for-github-docs/writing-content-to-be-translated}{Writing content to be translated}, orin Appendix~\ref{ch:translatable}, page~\pageref{ch:translatable}.
    \end{enumerate}
    \item \textbf{Consistent}: Documentation must maintain a consistent voice, tone, and style throughout, creating
    a cohesive experience for readers.
    \begin{enumerate}
        \item Ensure content adheres to the GitHub Docs style guide.
        For more information, see Style guide.
        \item Apply consistent terminology and naming conventions.
        \item Use branding elements (for example, product and feature names, logos, color schemes) consistently in
        the content.
    \end{enumerate}
\end{enumerate}

\subsection{Content Design Principles}\label{subsec:content_design}
We create product documentation that helps, teaches, and engages everyone who uses GitHub.
One step of this work is designing the content that we write.
We follow these principles when designing and planning content.

\begin{enumerate}
    \item Our content is user-centered and inclusive.
    We respect everyone who visits the docs and make content that works for them with our strategy, design, and style
    choices.
    \item Our content explains why our products are useful and helps people achieve their goals and priorities.
    \item We spend our resources creating high-quality, valuable documentation for GitHub’s community.
    \item We create just enough docs - more content makes everything more difficult to find, and anything added dilutes
    everything else (GitHub Zen).
    \item We iterate and ship to learn - as we learn more from experience, industry expertise, and working with our
    GitHub Docs community, we adjust our processes, practices, and guidelines.
\end{enumerate}

This content should also be designed provide a style guide and content models as building blocks and guidelines for
anyone to design and create documentation.

\begin{enumerate}
    \item Our style guide and content models apply to a range of scenarios.
    \item Decisions are based on what is best for people using our docs, not simply what is right or wrong according to
    grammar or style rules.
    We are flexible and open to change while maintaining consistency.
    \item We focus our attention on documenting high-impact, high-value scenarios rather than attempting to
    comprehensively cover every possible use case for the many GitHub products and features.
    \item Our highest priorities are clarity, meaning, correctness, and consistency.
    \item When making a style or structure decision, we consider what people are trying to do with the information and
    how our content can best support their goals.
    \item When a question specific to documentation is not covered by the style guide or content model, we evaluate it
    using these principles, then make a decision.
\end{enumerate}

\section{Best Practices for Writing GitHub Documentation}\label{sec:best_prac}
Whether you're creating a new article or updating an existing one, you should follow these guidelines given in the
Sub-Section below.
\subsection{Align Content to User Needs}\label{subsec:align_user}
Before you begin, it’s important to understand who you’re writing for, what their goals are, the core tasks or
concepts that the article will address, and what type of content to write.

In terms of defining the audience, the author must consider who will be reading the article, and what the reader is
trying to do with the information.

When defining the core purpose of the article, the author must answer a small number of questions.
Firstly, what should someone be able to do or understand after reading this article?
Choose one or two tasks or concepts that the content will discuss.
In addition, if there are additional tasks, concepts, or information that are not essential, consider if they can be
placed lower in the article, moved to another article, or omitted completely.

After defining the audience and cor purpose of the article, the author must also determine the type of content that
is presented in the article that they are writing.
These are given in the list below.
\begin{enumerate}
    \item Conceptual Content: Conceptual content helps people understand a feature or topic by providing a clear,
    high-level overview, explanation of how the feature or topic can help them on their journey, and context like
    use cases or examples.
    \item Referral Content: Referral Content is content that provides detailed information that people need while they
    are actively using a feature.
    Some major subjects may require their own referential article, especially if there
    is a large amount of referential content, such as for search syntax or YAML syntax in GitHub Actions.
    For smaller amounts of content or more specific information, like a list of a feature’s supported languages or
    hardware requirements, use referential sections in context within procedural or conceptual articles.
    \item Procedural Content: Procedural content gives context on how a task fits into someone's larger goal.
     This content helps people complete a task from start to finish while they are using GitHub.
     This content is also written as sections within larger articles.
    \item Troubleshooting Content: Troubleshooting content includes built-in errors we expect people to encounter,
    common problems reported to support, and situations people might encounter while completing tasks.
    Use troubleshooting sections in guides or procedural articles to keep solutions close to procedures.
    Work with support and product managers to surface common errors and include them in the documentation.
    \item Quickstart Content: Quickstarts enable people to quickly complete a discrete, focused task by illustrating
    a workflow with only essential steps, in about five minutes or 600 words.
    Quickstarts can be used for quickly getting set up with a new tool, or for quickly completing another task.
    For more complex tasks, use a tutorial.
    \item Tutorial Content: Tutorials are useful when someone has a basic understanding of the product and is
    interested in extending their understanding to solve a specific problem
\end{enumerate}

\section{Structure Content for Readability}\label{sec:struct_read}
Use the following best practices to structure the content.
When adding content to an existing article, follow the existing structure whenever possible.

\begin{enumerate}
    \item Provide initial context.
    Define the topic and state its relevance to the reader.
    \item Structure the content in a logical order by importance and relevance.
    Place information in order of
    priority, and in the order users will need it.
    \item Avoid long sentences and paragraphs.
    \begin{enumerate}
        \item Introduce concepts one by one.
        \item Use one idea per paragraph.
        \item Use one idea per sentence.
    \end{enumerate}
    \item Emphasize the most important information.
    \begin{itemize}
        \item Begin each sentence or paragraph with the most important words and takeaways.
        \item When explaining a concept, start with the conclusion, then explain it in more detail.
        (This is sometimes called an "inverted pyramid.")
        \item When explaining a complex topic, present readers with the basic information first, and disclose the
        details later in the article.
    \end{itemize}
    \item Use meaningful subheadings.
    Organize related paragraphs into sections.
    Give each section a subheading that is unique and that accurately describes the content.
    \item Consider using in-page links for longer content.
    This allows readers to jump to areas of interest and skip
    content that is irrelevant to them.
\end{enumerate}

\section{Write for Readability}\label{sec:readability}

The author must make it easy for busy users to read and understand the text.
To do so, follow the best practices below.

\begin{itemize}
    \item Use plain language.
    Use common, everyday words, and avoid jargon when possible.
    Terms that are well known to
    developers are fine, but don't assume that the reader knows the details of how GitHub works.
    \item Use active voice.
    \item Be concise.
    \begin{itemize}
        \item Write sentences that are simple and brief.
        \item Avoid complex sentences that contain multiple concepts.
        \item Pare down unnecessary details.
    \end{itemize}
\end{itemize}
For related information, see "Voice and tone" in \href{https://docs.github.com/en/contributing/style-guide-and-content-model/style-guide#voice-and-tone}{Style guide} and
\href{https://docs.github.com/en/contributing/writing-for-github-docs/writing-content-to-be-translated}{Writing content to be translated} in Appendix~\ref{ch:translatable}.

\section{Format for Scalability}\label{sec:scanability}

Most readers don't consume articles in their entirety.
Instead, they either \textit{scan} the page to locate specific
information, or \textit{skim} the page to get a general idea of the concepts.

When scanning or skimming content, readers skip over large chunks of text.
They look for elements that are related to their task or that stand out on the page, such as headings, alerts,
lists, tables, code blocks, visuals, and the first few words in each section.

Once the article has a clearly defined purpose and structure, you can apply the following formatting techniques to
optimize the content for scanning and skimming.
These techniques can also help to make content more understandable for all readers.

\begin{itemize}
    \item Use text highlighting such as boldface and hyperlinks to call attention to the most important points. Use text
    highlighting sparingly.
    Do not highlight more than $10\%$ of the total text in an article.
    \item Use formatting elements to separate the content and create space on the page.
    For example:
    \begin{itemize}
        \item Bulleted and numbered lists (with optional run-in subheads)
        \item Numbered lists
        \item \href{https://docs.github.com/en/contributing/style-guide-and-content-model/style-guide#alerts}{Alerts}
        \item Tables
        \item Visuals
        \item Code blocks and code annotations
    \end{itemize}
\end{itemize}


