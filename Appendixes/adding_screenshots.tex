\section{About Screenshots in GitHub Documentation}\label{sec:abt_ss_docu}
You can help users locate elements of the user interface that are difficult to find by adding screenshots to GitHub
documentation.
There are positives and negatives to adding a screenshot.
Screenshots make articles more visually scannable and make instructions easier to understand, especially for people
who have difficulty reading.
When supplied with alt text, screenshots help blind and low-vision users collaborate with sighted colleagues.

On the other hand, screenshots privilege sighted users, add length and load time to articles, and increase the
volume of content that needs to be maintained.
When captured at different pixel dimensions and degrees of zoom than the reader is using, screenshots can be confusing.
Therefore, we only add screenshots to GitHub Docs when they meet our criteria for inclusion.

\section{Criteria for Including a Screenshot}\label{sec:ss_inc_crit}
Use a screenshot to complement text instructions when an element of the user interface (UI) is hard to find, such as
in the following cases:

\begin{enumerate}
    \item The element is small or visually subtle: For example, Figure~\ref{fig:ss_yes_ui_small} displays an image
    of an instance where The edit button for a repository's social media preview image is small and visually
    unobtrusive.
    It may be hard to find among the other repository settings.
    \item The element is not immediately visible.
    For example, the element is contained in a drop-down menu.
    This can be seen in Figure~\ref{fig:ss_yes_hidden_ui}, where the options to clone a gist are contained under a
    dropdown menu labeled "Embed."

    \item The interface has multiple competing choices that can cause confusion.
    As seen in Figure~\ref{fig:ss_multi_choice}, There are three elements that could be interpreted as "settings" on the main page for a
    repository: the "Settings" tab, the gear icon in the "About" section of the right sidebar, and the account
    settings accessed via the profile picture.
\end{enumerate}

%Do not use screenshots for procedural steps where text alone is clear, or to show code commands or outputs.
% do we want this? idk if it would be helpful?
\begin{figure}
    \centering
    \caption{A UI element that may be hard to find without visual aid}
    \label{fig:ss_yes_ui_small}
    \includegraphics[width=0.75\textwidth]{Appendixes/Figures/screenshot-yes-ui-small}
\end{figure}

\begin{figure}
    \centering
    \caption{A UI element that is hidden until interacted with}
    \label{fig:ss_yes_hidden_ui}
    \includegraphics[width=0.75\textwidth]{Appendixes/Figures/screenshot-yes-ui-hidden}
\end{figure}

\begin{figure}
    \centering
    \caption{A UI has Multiple Competing Choices}
    \label{fig:ss_multi_choice}
    \includegraphics[width=1\textwidth]{Appendixes/Figures/screenshot-yes-ui-multi-choice}
\end{figure}

\section{Violations of Screenshot Inclusion Criteria}\label{sec:ss_no_follow_crit}
The following screenshots do not meet our criteria for inclusion, as seen in the list below.
\begin{enumerate}
    \item The UI element is easy to find: The "Create repository" button is visually prominent through size, color,
    and placement.
    There are few competing choices.
    This is seen in Figure~\ref{fig:ss_no_ui_easy}, where Text instructions are adequate to help the user complete
    the step.
    \item The UI has few, straightforward choices: It is simple and straightforward options, such as selecting or
    deselecting a checkbox, do not need a visual support.
    As seen in Figure~\ref{fig:ui_no_strfwd_choice}, Text instructions are adequate to help the user complete the step.
    However, There are also two accessibility implications of including the full sentence of text below the checkbox in
    the screenshot, as listed below:
    \begin{enumerate}
        \item The sentence is hard to read for low-sighted users, because it's small and not as crisp as HTML text.
        \item A person using a screen reader won't have access to the information, because it will not fit within
        alt text character limits.
        Including the text in the instructions would remedy this, but would be unnecessarily wordy.
    \end{enumerate}
\end{enumerate}

\begin{figure}
    \centering
    \caption{An easy-to-find UI element}
    \label{fig:ss_no_ui_easy}
    \includegraphics[width=1\textwidth]{Appendixes/Figures/screenshot-no-ui-easy-to-find}
\end{figure}

\begin{figure}
    \centering
    \caption{UI with straightforward choices}
    \label{fig:ui_no_strfwd_choice}
    \includegraphics[width=1\textwidth]{Appendixes/Figures/screenshot-no-ui-obv}
\end{figure}

\section{Requirements For Screenshots}\label{sec:ss_reqs}
In addition to the criteria for inclusion given in Section~\ref{sec:ss_inc_crit} and examples of violations of said
criteria in Section~\ref{sec:ss_no_follow_crit}, all screenshots must meet the following requirements as given below.

\subsection{Technical Specifications}\label{subsec:ss_tech_specs}
Images and photos should be given in a PNG file format, and are static images only (no GIFs).
These immages should be $144$ dpi (Dots per inch), $750$–$1000$ pixels wide for full-column images with a file size
of $250\text{KB}$ or less.
In addition, images must have descriptive file names, such as \verb|gist-embed-link.png| instead of \verb|right_side_page_03.png|\@.

Images captured on the macOS system must be retina images.
In Snagit and similar pieces of software, select Snagit $\to$ Preferences $\to$ Advanced and deselect "Scale down retina images
when sharing"

\subsection{Accessibility Requirements}\label{subsec:ss_access_reqs}
To meet the needs of more users, screenshots must obey the items outlined in the list below.

\begin{enumerate}
    \item Be accompanied by complete instructions in the procedural step, with no information conveyed entirely in
    visual form.
    \item Be full contrast, as in the interface itself, with nothing obscured or reduced in opacity or color contrast.
    \item Have alt text that describes the content of the image and the appearance of its highlighting, if any.
    For more information, see \href{https://docs.github.com/en/contributing/style-guide-and-content-model/style-guide#alt-text}{Style guide}.
    \item Be clear and crisp, with text and UI elements as legible as possible.
\end{enumerate}

\subsection{Visual Style}\label{subsec:vis_sty}

\begin{enumerate}
    \item Show a UI element with just enough surrounding context to help people know where to find the element on
    their screen.
    \item Reduce negative space by resizing your browser window until optimal.
    \item Show interfaces in light theme wherever possible.
    \begin{enumerate}
        \item For GitHub, select "Light default" in your appearance settings.
        For more information, see \href{https://docs.github.com/en/account-and-profile/setting-up-and-managing-your-personal-account-on-github/managing-personal-account-settings/managing-your-theme-settings}{Managing your theme settings}.
        \item For VSCode, select "GitHub light default" in the free \href{https://marketplace.visualstudio.com/items?itemName=GitHub.github-vscode-theme}{GitHub Theme} extension.
        \item If the software you need to screenshot is available in dark mode only, it's fine to use dark mode.
    \end{enumerate}
    \item If your username and avatar appear, replace them with \@ octocat's username and \href{https://avatars.githubusercontent.com/u/583231?v=4}{avatar}\@.
     Use the developer tools in your browser to replace your username with\@ octocat and to replace the URL of your
    avatar with \href{https://avatars.githubusercontent.com/u/583231?v=4}{octocat}.
    \item Do not include a cursor in the image.
\end{enumerate}

\subsubsection{Drop-Down Menus}\label{subsubsec:ss_dd_menus}
In many cases, there are two primary goals for showing a screenshot of a drop-down menu.
If the primary goal in showing a dropdown menu is to help the reader locate the menu itself, show the menu closed,
as seen in Figure~\ref{fig:ss_closed_dd} below.
\begin{figure}
    \centering
    \caption{Example of a closed dropdown menu}
    \label{fig:ss_closed_dd}
    \includegraphics{Appendixes/Figures/screenshot-closed-dd}
\end{figure}

If the primary goal in showing a dropdown menu is to help the reader distinguish among options within the menu, show
the menu open, as shown in Figure~\ref{fig:ss_open_dd}.
Capture open menus without focus (cursor or hover state).
Showing menu items with a white background ensures contrast with the dark orange outline, where present.

\begin{figure}
    \centering
    \caption{Example of a open dropdown menu}
    \label{fig:ss_open_dd}
    \includegraphics{Appendixes/Figures/screenshot-open-dd}
\end{figure}

\subsubsection{Highlighting in Screenshots}\label{subsubsec:hl_ss}

To highlight a specific UI element in a screenshot such as Figure~\ref{fig:ss_hl}, use our special theme for
\href{https://www.techsmith.com/snagit/}{Snagit} to apply a contrasting stroke around the element.

The stroke is the color \verb|fg.severe| in the
\href{https://primer.style/design/}{Primer Design System} (HEX #BC4C00 or RGB 188, 76, 0).
This dark orange has good color contrast on both white and black.
To check contrast on other background colors, use the
\href{https://www.tpgi.com/color-contrast-checker/}{Color Contrast Analyzer}.
\begin{figure}
    \centering
    \caption{Example of highlighting in a screenshot}
    \label{fig:ss_hl}
    \includegraphics{Appendixes/Figures/screenshot-hl}
\end{figure}





