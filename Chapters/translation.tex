Our documentation is translated into multiple languages.
How we approach writing the English language documentation can greatly improve the quality of those translations.

Use the following guidelines to ensure the content you create can be successfully translated.
For more information, see \href{https://docs.github.com/en/contributing/style-guide-and-content-model/style-guide}{Style guide}.

\begin{enumerate}
    \item Use examples that are generic and can be understood by most people.
    \item Avoid examples that are controversial or culturally specific to a group.
    For example, avoid 800 numbers and country specific addresses.
    If necessary, mention what countries the information applies to.
    \item Write in active voice.
    \item Write simple, short, and easy-to-understand sentences.
    \item Avoid using too many pronouns that can make text unclear.
    \item Avoid using slang and jokes.
    \item Avoid negative sentences.
    \item Use industry-standard acronyms whenever possible and explain custom acronyms.
    \item Use indicative mood.
    \item Eliminate redundant and wordy expressions.
    \item Avoid the excessive use of stacked modifiers (noun strings).
    The translator can misunderstand which noun is the one being modified.
    Lots of stacked modifiers can lead to incorrect translations because it's not easy to determine what modifies what.
    For example, use \enquote{Default source settings for the public repository}  instead of \enquote{public repository default source settings.}
    \item Avoid invisible plurals in which it is not clear if the first noun is meant to be singular or plural.
    For example, when using the term \enquote{file retrieval,}  it's unclear if you are retrieving one file or all files.
    Provide more context to eliminate ambiguity.
    In the example given, you can use \enquote{retrieving all the files} or \enquote{retrieving the source.md file.}
    \item Avoid nominalization.
    For example, instead of \enquote{to reach a conclusion,}  use \enquote{conclude.}
    \item Avoid using ambiguous modal auxiliary verbs.
    Avoid using words such as \enquote{may}  and \enquote{might.}  Be clearer to avoid ambiguity.
    \item Avoid gender-specific words.
    \item Avoid prepositional phrases.
    Instead of writing \enquote{after trying many times} or \enquote{according to the repository log,}  write more directly.
    For example, \enquote{after trying three times.}
    \item Avoid vague nouns and pronouns (vague sentence subject).
    Vague nouns and pronouns can make it unclear who or what you are referring to, especially when that content has to be translated.
    For example, \enquote{Maintainers and contributors have access to files and comments.
    In the pull request they make changes to it.}  In this sentence, it is not clear if the changes are being made to the file or comments.
    If a pronoun seems to refer to more than one antecedent, either reword the sentence to make the antecedent clear or replace the pronoun with a noun to eliminate ambiguity.
    \item Keep inline links to a minimum.
    If they are necessary, preface them with a phrase such as \enquote{For more information, see \textit{Link title.}}
    Alternatively, add relevant links to a \enquote{Further reading}  section at the end of the topic.
\end{enumerate}

