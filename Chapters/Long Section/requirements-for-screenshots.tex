In addition to the criteria for inclusion given in Section~\ref{sec:ss_inc_crit} and examples of violations of said
criteria in Section~\ref{sec:ss_no_follow_crit}, all screenshots must meet the following requirements as given below.

\subsection{Technical Specifications}\label{subsec:ss_tech_specs}
Images and photos should be given in a PNG file format, and are static images only (no GIFs).
These immages should be $144$ dpi (Dots per inch), $750$–$1000$ pixels wide for full-column images with a file size
of $250\text{KB}$ or less.
In addition, images must have descriptive file names, such as \verb|gist-embed-link.png| instead of \verb|right_side_page_03.png|\@.

Images captured on the macOS system must be retina images.
In Snagit and similar pieces of software, select Snagit $\to$ Preferences $\to$ Advanced and deselect \enquote{Scale down retina images
when sharing}

\subsection{Accessibility Requirements}\label{subsec:ss_access_reqs}
To meet the needs of more users, screenshots must obey the items outlined in the list below.

\begin{enumerate}
    \item Be accompanied by complete instructions in the procedural step, with no information conveyed entirely in
    visual form.
    \item Be full contrast, as in the interface itself, with nothing obscured or reduced in opacity or color contrast.
    \item Have alt text that describes the content of the image and the appearance of its highlighting, if any.
    For more information, see \href{https://docs.github.com/en/contributing/style-guide-and-content-model/style-guide#alt-text}{Style guide}.
    \item Be clear and crisp, with text and UI elements as legible as possible.
\end{enumerate}

\subsection{Visual Style}\label{subsec:vis_sty}

\begin{enumerate}
    \item Show a UI element with just enough surrounding context to help people know where to find the element on
    their screen.
    \item Reduce negative space by resizing your browser window until optimal.
    \item Show interfaces in light theme wherever possible.
    \begin{enumerate}
        \item For GitHub, select \enquote{Light default}  in your appearance settings.
        For more information, see \href{https://docs.github.com/en/account-and-profile/setting-up-and-managing-your-personal-account-on-github/managing-personal-account-settings/managing-your-theme-settings}{Managing your theme settings}.
        \item For VSCode, select \enquote{GitHub light default}  in the free \href{https://marketplace.visualstudio.com/items?itemName=GitHub.github-vscode-theme}{GitHub Theme} extension.
        \item If the software you need to screenshot is available in dark mode only, it's fine to use dark mode.
    \end{enumerate}
    \item If your username and avatar appear, replace them with \@ octocat's username and \href{https://avatars.githubusercontent.com/u/583231?v=4}{avatar}\@.
     Use the developer tools in your browser to replace your username with\@ octocat and to replace the URL of your
    avatar with \href{https://avatars.githubusercontent.com/u/583231?v=4}{octocat}.
    \item Do not include a cursor in the image.
\end{enumerate}

\subsubsection{Drop-Down Menus}\label{subsubsec:ss_dd_menus}
In many cases, there are two primary goals for showing a screenshot of a drop-down menu.
If the primary goal in showing a dropdown menu is to help the reader locate the menu itself, show the menu closed,
as seen in Figure~\ref{fig:ss_closed_dd} below.
\begin{figure}
    \centering
    \caption{Example of a closed dropdown menu}
    \label{fig:ss_closed_dd}
    \includegraphics{Chapters/Figures/screenshot-closed-dd}
\end{figure}

If the primary goal in showing a dropdown menu is to help the reader distinguish among options within the menu, show
the menu open, as shown in Figure~\ref{fig:ss_open_dd}.
Capture open menus without focus (cursor or hover state).
Showing menu items with a white background ensures contrast with the dark orange outline, where present.

\begin{figure}
    \centering
    \caption{Example of an open dropdown menu}
    \label{fig:ss_open_dd}
    \includegraphics{Chapters/Figures/screenshot-open-dd}
\end{figure}

\subsubsection{Highlighting in Screenshots}\label{subsubsec:hl_ss}

To highlight a specific UI element in a screenshot such as Figure~\ref{fig:ss_hl}, use our special theme for
\href{https://www.techsmith.com/snagit/}{Snagit} to apply a contrasting stroke around the element.

The stroke is the color \verb|fg.severe| in the
\href{https://primer.style/design/}{Primer Design System} (HEX #BC4C00 or RGB 188, 76, 0).
This dark orange has good color contrast on both white and black.
To check contrast on other background colors, use the
\href{https://www.tpgi.com/color-contrast-checker/}{Color Contrast Analyzer}.
\begin{figure}
    \centering
    \caption{Example of highlighting in a screenshot}
    \label{fig:ss_hl}
    \includegraphics{Chapters/Figures/screenshot-hl}
\end{figure}