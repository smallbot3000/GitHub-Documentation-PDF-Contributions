At GitHub, we strive to create documentation that is accurate, valuable, inclusive, accessible, and easy to use.
Before contributing to GitHub Docs, please take a moment to familiarize yourself with GitHub's documentation philosophy,
fundamentals, and content design principles, given in Sub-Sections.

\subsection{GitHub's Documentation Philosophy}\label{subsec:doc_ply}
Our documentation philosophy guides what content we create and how we create it.
As the home to the world's largest community of developers, we want to make sure that our documentation is accurate,
valuable, inclusive, and easy to use.
Our documentation philosophy leads us toward these goals.

\begin{enumerate}
    \item We advocate for our users.
    This can occur at any point in the documentation process, from planning to writing to publishing.
    We respond to feedback and proactively work to create the best user experience possible on the GitHub site.
    \item We write for an international audience, so that our content can be translated and is inclusive of all people.
    \item We create content that can be accessed by a broad group of users.
    Our content toolkit includes resources and guidance to reduce barriers for people with disabilities, and we
    prioritize keeping up to date with the latest accessibility standards.
    \item We create task-based content.
    We consider what people are trying to accomplish when they use GitHub, and then we create content that helps
    them achieve their goals and discover new possibilities.
    \item We collaborate with teams across GitHub and the open source community to create high-quality content.
    \item We continually learn and improve to curate the best experience for people using GitHub.
\end{enumerate}
%We may want examples of each or most of these points

\subsection{GitHub's Documentation Fundamentals}\label{subsec:doc_fundm}
All content published must meet a small number of fundamental requirements.
Use the lists below to help ensure your contributions are accurate, accessible and inclusive, and consistent.
\begin{enumerate}
    \item \textbf{Accurate}: Documentation must be correct and accurate.
    Ensure that the content is free from factual, spelling, and formatting errors.
    \item Documentation is up to date with the latest accessibility standards, and is written to be inclusive and
    translation-friendly.
    \begin{enumerate}
        \item Ensure content adheres to the accessibility and screenshot guidelines.
        For more information, see \href{https://docs.github.com/en/contributing/writing-for-github-docs/creating-screenshots}{Creating screenshots}, or in Appendix~\ref{ch:create_ss}, page~\pageref{ch:create_ss}.
        \item Ensure content can be successfully translated.
        For more information, see \href{https://docs.github.com/en/contributing/writing-for-github-docs/writing-content-to-be-translated}{Writing content to be translated}, orin Appendix~\ref{ch:translatable}, page~\pageref{ch:translatable}.
    \end{enumerate}
    \item \textbf{Consistent}: Documentation must maintain a consistent voice, tone, and style throughout, creating
    a cohesive experience for readers.
    \begin{enumerate}
        \item Ensure content adheres to the GitHub Docs style guide.
        For more information, see Style guide.
        \item Apply consistent terminology and naming conventions.
        \item Use branding elements (for example, product and feature names, logos, color schemes) consistently in
        the content.
    \end{enumerate}
\end{enumerate}

\subsection{Content Design Principles}\label{subsec:content_design}
We create product documentation that helps, teaches, and engages everyone who uses GitHub.
One step of this work is designing the content that we write.
We follow these principles when designing and planning content.

\begin{enumerate}
    \item Our content is user-centered and inclusive.
    We respect everyone who visits the docs and make content that works for them with our strategy, design, and style
    choices.
    \item Our content explains why our products are useful and helps people achieve their goals and priorities.
    \item We spend our resources creating high-quality, valuable documentation for GitHub’s community.
    \item We create just enough docs - more content makes everything more difficult to find, and anything added dilutes
    everything else (GitHub Zen).
    \item We iterate and ship to learn - as we learn more from experience, industry expertise, and working with our
    GitHub Docs community, we adjust our processes, practices, and guidelines.
\end{enumerate}

This content should also be designed provide a style guide and content models as building blocks and guidelines for
anyone to design and create documentation.

\begin{enumerate}
    \item Our style guide and content models apply to a range of scenarios.
    \item Decisions are based on what is best for people using our docs, not simply what is right or wrong according to
    grammar or style rules.
    We are flexible and open to change while maintaining consistency.
    \item We focus our attention on documenting high-impact, high-value scenarios rather than attempting to
    comprehensively cover every possible use case for the many GitHub products and features.
    \item Our highest priorities are clarity, meaning, correctness, and consistency.
    \item When making a style or structure decision, we consider what people are trying to do with the information and
    how our content can best support their goals.
    \item When a question specific to documentation is not covered by the style guide or content model, we evaluate it
    using these principles, then make a decision.
\end{enumerate}